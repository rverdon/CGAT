% Paper for the ChouchBase assignment.

\documentclass[]{IEEEtran}

\usepackage{float}
\usepackage{url}
\usepackage{graphicx}
\usepackage{color}

\begin{document}

\title{No Dogs On The Couch$_{Base}$}

% author names and affiliations
% use a multiple column layout for up to two different
% affiliations

\author{\IEEEauthorblockN{Eriq Augustine, Ryan Hnarkis, Aldrin Montana, Ryan Verdon, Tyler Yero}
\\
\IEEEauthorblockA{Department of Computer Science\\
Cal Poly, San Luis Obispo\\
 \textsf{\{eaugusti, rhnaraki, amontana, rverdon, tyero\}@calpoly.edu}
}
}

\maketitle

\thispagestyle{empty}
\pagestyle{empty}

\section{Introduction}
Our team decided to compare a MySQL cluster versus a Couchbase cluster for running a
web application. We are using an eight table SQL schema from Ryan Verdon's Thesis. To evaluate
the performance we autogenerated around 400MB of data and inserted the same data into each cluster.
Obviously using a different data model for Couchbase to accomodate for the fact it is a NoSQL key-value store
instead of a relational database. We then created a set of six experiments to run. Five came from common 
use cases found in the web application. While the other was designed to test the performance on each cluster
of repeatedly writing and reading from the same object. Our major metrics are write performance, read performance,
and time-delay between writes and reads of the same data. The rest of the paper is broken down as follows.
Section 2 describes in depth how we designed the experiment. Section 3 discusses our implementation. Section 4 goes
into the results of each of our experiments and some analysis on the data we obtained. Lastly, section 5 presents 
our conclusions.

\section{Experiment Design}
TODO

\section{Workloads}
The following workloads were applied to the both the MySQL and Couchbase clusers.
\subsection{Assigning a Task}
TODO
\subsection{Profiling}
TODO
\subsection{Read and Writing Performance}
TODO
\subsection{Publishing}
TODO
\subsection{Adding and Removing Group Membership}
Adding or removing users to a group is a task common for classroom situations. In MySQL, this operation requires a join between three tables: Groups, Users, and GroupMembership. In Couchbase, this requires retrieving the key for the desired group and removing the user from the members field.

\section{Implementation}
TODO

\section{Evaluation}
TODO

\section{Conclusions}
TODO

% TODO: Uncomment if we use any refs
% \bibliographystyle{acm}
% \bibliography{refs}

\end{document}
